\documentclass[11pt,a4paper]{article}
\usepackage{moesioa}
\usepackage{sagetex}
\usepackage{caption}
\usepackage{amsmath}
%\usepackage[amsmath]{maxiplot}
\usepackage{csquotes}
\usepackage[url,doi,style=numeric,backend=bibtex]{biblatex}
\addbibresource{referencia.bib}
\newcommand{\nomet}{}
\newcommand{\nome}{}
\newcommand{\nomer}{\bf Aplicações  Latex -- Sagemath}
\newcommand{\titu}{}
\newcommand{\disc}{}
\newcommand{\curso}{}
\newcommand{\inst}{$(1+i)^{t}$}
\newcommand{\instr}{\today}
%================================================================
\begin{document}
\Large
%===============================================================
\large
\begin{center}
\noindent  \nomer
\end{center}
\hfill   %\today\\[2mm]
\hrule\ 
%==========================================================================================
\section*{\sc Problemas}
Considere o problema de\cite{fina2007mathias}:
 


{\bf Questão: }Se uma financeira apresentar o coeficiente de $0,09749$ para $12$ prestações mensais e além disso cobrar $2\%$ sobre o valor financiado, a título de despesas administrativas (desconto este que será feito no ato da compra), qual será a taxa de juros efetiva?

{\sol Temos que}

\ben
cf=\frac{i}{1-(1+i)^{-12}}=0,09749\\
\een
Considerando a equação
\ben
a(i)= \frac{1-(1+i)^{-12}}{i}
\een
\begin{sagesilent}
a(i)= (1-(1+i)^-12)/i
\end{sagesilent}

\sageplot[width=.5\textwidth]{plot(a(i), i, 0, 5)}

Desenvolvendo o problema:

\ben
a(0,027)=\sage{a(0.027)}\\
a(0.03)=\sage{a(0.03)}\\
%a(0.09)=\sage{a(0.09)}\\
%a(0.05)=\sage{a(0.05)}
\een

%==========================================================================================
%==========================================================================================
\printbibliography
%==========================================================================================
\end{document}
